Definitions of SOH:
Currently, numerous definitions of battery SOH exist in the literature\cite{s41598-025-92262-8}. Mainstream evaluation methods typically rely on changes in capacity or internal resistance\cite{electronics-13-01675}.
\begin{itemize}
    \item \textbf{Capacity Based Definition:} The most widely used approach defines SOH based on the degree of capacity degradation \cite{s41598-025-92262-8}. It is calculated as the percentage of the current maximum available capacity ($Q_m$) of the battery to its rated capacity ($Q_r$)\cite{electronics-13-01675}\cite{wevj-12-00113}: $$SOH = \frac{Q_m}{Q_r} \times 100\% \quad (1)$$ The change of capacity is directly related to the SOH of the battery, which can clearly and stably reflect the degradation process \cite{s41598-025-92262-8}. A battery is generally considered at the end of life (EOL) when its capacity has decreased to 80\% of its original value, as stated by IEEE standard 1188.1996\\ \cite{wevj-12-00113}\cite{energies-18-00342-v2}.
    \item \textbf{Internal Resistance Based Definition:}  SOH can also be defined according to the internal resistance of the battery\cite{wevj-12-00113}: $$SOH = \frac{R_e - R}{R_e - R_n} \times 100\% \quad (2)$$ where $R$ is the internal resistance under the current state, $R_e$ is the internal resistance of the battery when it reaches the end of life, and $R_n$ is the internal resistance of the new battery\cite{wevj-12-00113}. The increase of internal resistance is an essential indicator of battery aging and contributes to the further decline of battery SOH\cite{wevj-12-00113}.
\end{itemize}

Relationship with other battery states:
\begin{itemize}
    \item \textbf{State of Charge (SOC):} Accurate assessment of battery cell capacity, which is reflected in the SOH, ensures timely replacement. Cell health (SOH) is evaluated by comparing initial and current capacity. The SOC is crucial for the SOH calculation\cite{energies-13-01811-v2}. Joint estimation of SOH and SOC is critical, with capacity-based and resistance-based SOH estimation processes playing a significant role in updating SOC estimation\cite{energies-13-01811-v2}.
    \item \textbf{Remaining Useful Life (RUL):} SOH estimation is a precondition for estimating the battery's remaining useful life (RUL)\cite{wevj-12-00120-v2}. While some studies focus on RUL separately, most combine RUL and SOH, where SOH is predicted first, followed by RUL based on the SOH\cite{batteries-10-00181-v2}.
\end{itemize}

As SOH estimation cannot be measured directly during regular operation, one cannot easily measure full capacity without fully discharging the battery. A variety of estimation techniques have been developed. There are three primary categories of methods for estimating the SOH of lithium-ion batteries: model-based methods, data-driven methods, and hybrid (fusion) methods\cite{wevj-12-00113}.
\begin{itemize}
    \item \textbf{Electrochemical Model-Based Estimation:} These approaches use physics-based electrochemical models of the battery to predict aging behavior. These methods rely on understanding lithium-ion batteries' degradation and failure mechanisms to estimate and predict SOH. These methods typically involve developing mathematical models that describe the battery's internal electrochemical and physical processes \cite{electronics-13-01675}\cite{s41598-025-92262-8}. The accuracy of the SOH estimation depends on how well the model's key parameters represent the internal aging of the battery \cite{wevj-12-00113}. Model-based methods for battery analysis offer several advantages, including their ability to provide detailed explanations of discharge behavior and underlying degradation mechanisms, with electrochemical models yielding valuable insights into the battery's internal state, while equivalent circuit models (ECMs) stand out for their simplicity and strong engineering practicality \cite{electronics-13-01675}\cite{wevj-12-00113}. However, these methods also face notable drawbacks, such as the complexity of many models, which involve numerous parameters that are challenging to identify accurately, and the susceptibility of ECM parameter accuracy to environmental changes, resulting in fluctuating outcomes and accumulating errors \cite{electronics-13-01675}\cite{wevj-12-00113}. Additionally, these approaches often incur high computational costs, restricting their use in real-time applications, and exhibit weak anti-interference capabilities in practical settings, where achieving high accuracy remains difficult. Furthermore, ECMs may overlook subtle factors influencing state-of-health (SOH) attenuation and fail to comprehensively address complex external stress variations, limiting their overall effectiveness \cite{electronics-13-01675}\cite{wevj-12-00113}.
    \item \textbf{Empirical Model-Based Estimation:} Empirical or equivalent-circuit models (ECM) are widely used in practice due to their simplicity \cite{energies-13-02825-v2}. In this method, the battery is modeled as an electric circuit (combinations of voltage sources, resistors, and capacitors) that imitates the battery's charge/discharge behavior. Aging of the battery is reflected when there are changes in the model parameters—for instance, the battery's open circuit voltage curve, internal resistance, or capacitance values. By tracking how these parameters drift over time, the BMS can infer SOH. Some of the generally used techniques involve estimation via filtering: the BMS uses algorithms like Kalman filters to recursively update model parameters (e.g., internal resistance increasing with age) based on measurements \cite{wevj-12-00113}. For example, an Extended Kalman Filter (EKF) or Unscented Kalman Filter (UKF) can be employed to jointly estimate the battery’s SOC and capacity (treating capacity as an unknown parameter that slowly changes) \cite{wevj-12-00113}. Empirical models are computationally lightweight and can be tuned to cell data, so most commercial BMS rely on them \cite{energies-13-02825-v2}. Their drawback is that they may not capture all failure modes; they assume that past observed trends (e.g., capacity fade vs. cycle count) will continue, which might not hold under new conditions.
    \item \textbf{Data-Driven Methods:} Data-driven methods estimate SOH by analyzing data generated during battery operation, such as voltage, current, temperature, and charge-discharge duration, using artificial intelligence or statistical models \cite{wevj-12-00113}\cite{energies-13-01811-v2}\cite{electronics-13-01675}\cite{energies-13-02825-v2}.\\ These are often referred to as "black-box" models because they do not necessarily require an understanding of the internal physical or chemical reactions within the battery \cite{electronics-13-01675}. The relationship between measured parameters and SOH is learned from historical data \cite{wevj-12-00113}\cite{electronics-13-01675}. However, data-driven methods typically need large datasets of battery aging under various conditions for training, and their extrapolation beyond the training domain is not guaranteed. Ensuring robustness and interpretability of purely data-driven SOH estimates (especially in safety-critical applications like EVs) remains an active research area.
    \item \textbf{Hybrid (fusion) Methods:}\cite{electronics-13-01675}\cite{energies-13-01811-v2} Hybrid (fusion) methods combine two or more different SOH estimation techniques, often integrating model-based and data-driven approaches, to improve accuracy, robustness, and reliability \cite{wevj-12-00113}. The aim is to leverage the strengths of different methods while compensating for their limitations.
\end{itemize}