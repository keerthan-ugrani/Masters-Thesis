This research successfully addressed the critical challenge of accurate State of Health (SOH) estimation for lithium-ion batteries in dynamic operational environments, where traditional semi-empirical models often falter due to their reliance on static parameters and inability to account for nuanced degradation factors like variable loading, temperature fluctuations, and intermittent stress cycles. The Adaptive Multi-Horizon SOH Correction Algorithm emerged as a robust solution, reducing the baseline semi-empirical model’s RMSE from 5.24\% to an average RMSE\_ground\_truth of 1.24\% across eight batteries, with standout performance on short datasets like Battery B0028 (0.38\%) and challenges with noisy data in Battery B0018 (3.61\%). A key finding is the algorithm’s ability to adapt to diverse degradation patterns—capturing rapid drops in B0006 and stable trends in B0007—through a novel multi-step pipeline integrating Moving Horizon Estimation (MHE), linear regression stacking, and Extended Kalman Filter (EKF) smoothing. This approach mitigated the cumulative estimation errors inherent in semi-empirical models by introducing adaptive correction factors ($k_1$, $k_2$), supported by Huber Loss for noise robustness, and enabled real-time correction of SOH deviations. The preprocessing functions (load\_mat\_file, extract\_discharge\_data, extract\_charge\_data) streamlined data handling, transforming raw .mat files into actionable DataFrames, enhancing efficiency for battery health monitoring.

The developed framework, termed NMOps (Numerical Model Operations), generalizes the architecture to any numerical model, not just semi-empirical ones, by leveraging an MLOps-like structure that supports automated validation, deployment, and maintenance. NMOps, built on MLOps best practices, is scalable and manageable, facilitating cloud-based deployment across heterogeneous battery fleets through parallel processing with Dask and potential distributed computing with Spark. The algorithm acts as a core component of the NMOps architecture, integrating the semi-empirical model into this MLOps-like framework by providing a continuous correction mechanism and automated validation against test-bench data, addressing the absence of such mechanisms in traditional models. This enables more efficient SOH estimation for EV battery management, supporting predictive maintenance and lifecycle optimization with improved accuracy, particularly for batteries with limited cycles or predictable degradation patterns, and enhances model interpretability and scalability for fleet-wide applications.

\section{Future Scope}
Several avenues for future work are proposed to address the identified limitations and further enhance the Adaptive Multi-Horizon SOH Correction Algorithm. Incorporating additional features such as temperature, voltage, and current measurements into the MHE correction process could improve accuracy, particularly for noisy datasets like B0018, by providing more context for degradation patterns. Advanced noise-handling techniques, such as anomaly detection using Isolation Forests or autoencoders, could mitigate the impact of extreme noise, addressing cases where the semi-empirical model’s imperfections lead to significant discrepancies (e.g., B0018).

Additionally, replacing the current linear correction factors and fixed window sizes with neural network-based approaches offers significant potential. Neural Ordinary Differential Equations (Neural ODEs) could model the continuous-time dynamics of battery degradation more accurately, replacing the discrete window-based MHE corrections with a differential equation solver that learns the degradation trajectory from data. This would allow the algorithm to capture intermediate degradation patterns missed by the 3- and 10-cycle windows, potentially reducing RMSE for batteries like B0031. Neural Partial Differential Equations (Neural PDEs) could further extend this by modeling spatial-temporal degradation effects (e.g., temperature gradients across the battery), incorporating physical constraints into the correction process. These neural approaches can be integrated with machine learning algorithms, such as Long Short-Term Memory (LSTM) networks for temporal modeling of SOH trends, or Gradient Boosting models like XGBoost for stacking, replacing the current linear regression to handle non-linear relationships better. For instance, an LSTM could predict SOH sequences, feeding into a Neural ODE for continuous correction, while XGBoost could stack the predictions with additional features. These enhancements, supported by GPU acceleration for faster training, could achieve higher accuracy and scalability, paving the way for real-time SOH correction in EV fleet management systems.