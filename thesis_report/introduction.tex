
% This file contains the Introduction of the thesis

A Battery Management System (BMS) is a core component in modern battery-powered applications, designed to oversee and regulate the operation of battery packs to ensure safety, reliability, and performance \cite{wevj-12-00120-v2}. Battery packs typically consist of multiple electrochemical cells connected in series and/or parallel, each exhibiting variations in behavior that necessitate precise management to enhance longevity and reduce operational costs \cite{energies-18-00342-v2,wevj-12-00120-v2}. Maintaining optimal operating conditions, such as limiting temperatures, controlling charge/discharge rates, and enforcing voltage and current boundaries, is essential to delay degradation and avoid failures.

The BMS operates as an embedded control unit within the battery system. It performs various critical functions, including real-time monitoring of thermal and electrical parameters, balancing cell voltages, estimating key performance indicators like State of Charge (SOC) and State of Health (SOH), and executing protective actions when anomalies are detected. These tasks collectively improve battery efficiency, extended service life, and safer operation. Furthermore, the BMS interfaces with other vehicle or system controllers—such as propulsion systems, thermal management units, communication modules, and diagnostic tools—to support seamless integration and operational coordination \cite{wevj-12-00120-v2}.


In Electric Vehicles (EVs), the Battery Management System (BMS) is tasked with managing the inherent complexities of lithium-ion batteries, which include high energy density, substantial power requirements, broad operating temperature ranges, and exposure to demanding environmental and driving conditions. Accurate estimation of state variables—particularly the State of Charge (SOC) and the State of Health (SOH)—is essential for ensuring efficient and reliable battery operation \cite{105207_1_5.0172683}. These indicators offer critical insights into the current condition and capacity of the battery, forming the foundation for effective energy management strategies.

The SOH, in particular, serves as a measure of how much the battery has deteriorated from its original state \cite{s41598-025-92262-8}. It reflects the capacity loss and internal resistance increase, which are key factors in assessing the long-term usability of the battery. Timely and precise SOH estimation supports predictive maintenance and enhances the overall safety and longevity of EV systems \cite{105207_1_5.0172683,electronics-13-01675}. Moreover, a well-calibrated SOH model can guide operational decisions and replacement schedules, minimizing unexpected failures and contributing to the sustainable deployment of electric mobility solutions \cite{electronics-13-01675}.


\subsection*{Significance of Battery State of Health (SOH)}

The State of Health (SOH) is a critical parameter in battery diagnostics and lifecycle management. Its accurate assessment offers multiple operational and strategic advantages:

\begin{itemize}
    \item \textbf{Efficient Battery Utilization:} Knowledge of the SOH allows for more informed decisions regarding charge/discharge control and thermal management, enhancing overall system efficiency \cite{105207_1_5.0172683}.
    
    \item \textbf{Lifecycle Optimization:} By enabling early detection of degradation trends, reliable SOH prediction supports the development of optimized usage patterns and maintenance protocols that extend battery longevity \cite{electronics-13-01675}.
    
    \item \textbf{Safety Enhancement:} Predictive SOH monitoring helps prevent hazardous situations by identifying potential failure points in advance, thereby contributing to safer operation of electric vehicles and storage systems \cite{electronics-13-01675}.
    
    \item \textbf{Maintenance Planning and Cost Reduction:} SOH insights facilitate timely servicing and component replacement, reducing the risk of unexpected breakdowns and minimizing downtime and maintenance costs \cite{105207_1_5.0172683}.
    
    \item \textbf{Remaining Useful Life (RUL) Estimation:} A well-calibrated SOH model\\ provides the foundation for estimating the Remaining Useful Life (RUL), \\ which is essential for long-term planning and predictive asset management\\ \cite{wevj-12-00120-v2}.
\end{itemize}


\subsection*{Key Factors Influencing Battery SOH}

The degradation of lithium-ion batteries, and thus their State of Health (SOH), is governed by several operational, environmental, and material-related factors. These influences can vary significantly depending on usage patterns and system design:

\begin{itemize}
    \item \textbf{Environmental and Electrical Conditions:} Parameters such as operating temperature, charging/discharging current, depth of discharge (DOD), and the maintained state of charge (SOC) directly affect the rate of aging and performance deterioration \cite{energies-18-00342-v2}.
    
    \item \textbf{Charge-Discharge Cycling:} Repeated cycling results in gradual capacity fade and resistance increase. Conditions such as deep discharges, overcharging, or operation at extreme SOC levels (below 25\% or above 90\%) have been shown to accelerate degradation more rapidly than moderate cycling behavior \cite{electronics-13-01675,batteries-10-00181-v2,energies-18-00342-v2}. Elevated temperatures exacerbate these effects, irrespective of the cell chemistry \cite{energies-18-00342-v2}.
    
    \item \textbf{Calendar Aging:} Even in the absence of active cycling, time-dependent chemical processes contribute to internal degradation, particularly under elevated storage temperatures \cite{energies-18-00342-v2}.
    
    \item \textbf{Material Decomposition and Structural Changes:} Long-term usage leads to physical and chemical alterations in the electrodes and electrolyte, such as lithium plating, SEI (solid electrolyte interphase) layer growth, and structural breakdown, all of which degrade SOH \cite{energies-18-00342-v2}.
\end{itemize}


Although the previously discussed factors contribute significantly to understanding battery degradation, they are not exhaustive. Several challenges remain in accurately modeling real-world battery behavior using semi-empirical approaches. The key limitations include:

\begin{itemize}
    \item \textbf{Sensitivity to Noisy and Non-Stationary Data:} Semi-empirical models often lack mechanisms to address sudden shifts or irregularities in measurement data. These unexpected variations, caused by sensor inaccuracies, environmental disturbances, or transient events, can lead to significant deviations in SOH predictions, as these models do not inherently adapt to changing data distributions \cite{s21248304}.
    
    \item \textbf{Dependence on Fixed Parameterization:} Typically, these models rely on parameters calibrated under specific laboratory or historical conditions, such as degradation coefficients derived from controlled cycling experiments\\ \cite{doi:10.1021/ie990486w}. When deployed in dynamic or unseen environments, the fixed nature of these parameters limits generalizability, often resulting in performance degradation.
    
    \item \textbf{Lack of Real-Time Feedback and Learning Mechanisms:} Conventional semi-empirical frameworks operate statically and do not support continuous learning or model adaptation. In cloud-based or time-evolving deployment scenarios, this limitation restricts the ability to respond to system drift, thereby reducing accuracy and operational efficiency over time \cite{7462263,8633871}.
\end{itemize}

The widespread adoption of lithium-ion batteries across high-demand applications, such as electric vehicle (EV) fleets and utility-scale energy storage systems, has emphasized the critical need for reliable and scalable models to estimate the State of Health (SOH). Accurate SOH estimation is essential for optimizing battery utilization, ensuring system safety, and extending service life within battery management systems (BMS). Among various modeling approaches, semi-empirical models have gained popularity due to their favorable trade-off between interpretability and computational efficiency, as they combine physics-informed equations with empirically derived parameters.

However, as deployment environments become more dynamic and data-intensive, several structural limitations of semi-empirical models become evident, particularly concerning adaptability, scalability, and lifecycle integration in cloud-based or distributed architectures.

\begin{itemize}
    \item \textbf{Limited Representation of Degradation Phenomena:} Battery aging is governed by a complex interplay of environmental and usage factors, including temperature variation, depth of discharge, cycling intensity, and microstructural changes within electrodes. Semi-empirical models often simplify or omit such detailed dynamics, which can lead to underestimation of degradation during specific operating phases, such as early-cycle structural transitions in electrode materials.
    
    \item \textbf{Lack of Continuous Validation and Feedback:} These models typically do not incorporate systematic validation mechanisms that compare predicted SOH against experimental or test-bench benchmarks. The absence of automated feedback loops hinders performance monitoring and error correction over time, ultimately reducing the trust and transparency required for critical systems.
    
    \item \textbf{Poor Integration with MLOps and Cloud Infrastructures:} Traditional semi-empirical frameworks were not developed with modern deployment practices in mind. They lack compatibility with Machine Learning Operations (MLOps) pipelines, which support automated model retraining, deployment, and monitoring. This architectural gap makes it difficult to scale such models across cloud environments or distributed battery systems that rely on real-time data ingestion and continuous updates.
    
    \item \textbf{Scalability and Computational Bottlenecks:} While semi-empirical models are suitable for embedded or low-resource contexts, extending their use to cloud-scale applications, where large datasets must be processed concurrently, demands significant architectural enhancements. Their current form does not readily support distributed computing or modular scaling.
\end{itemize}

\subsection*{Identified Research Gaps in SOH Estimation Frameworks}

Recent literature reveals several persistent challenges in the lifecycle management of semi-empirical models, particularly when applied to cloud-based or large-scale deployments. These gaps highlight the need for more adaptable and scalable frameworks for State of Health (SOH) estimation:

\begin{itemize}
    \item \textbf{Insufficient Real-Time Processing Capability:} Although many data-driven approaches have been explored for SOH estimation, most are constrained to offline processing workflows. This limitation restricts their applicability in time-sensitive environments where real-time predictions are essential. Studies on edge-cloud collaboration indicate that current fusion models are often computationally intensive and fail to meet the latency requirements of dynamic, real-world systems.
    
    \item \textbf{Lack of Integration with Modern MLOps Pipelines:} Existing semi-empirical models are not readily compatible with Machine Learning Operations (MLOps) infrastructures, which are critical for automating deployment, monitoring, retraining, and validation in production environments. The absence of such integration impedes continuous model adaptation and undermines the robustness of SOH estimation in evolving operational contexts.
    
    \item \textbf{Limited Support for Hybrid and Multi-Scenario Modeling:} Battery applications vary widely across domains, from electric vehicles to grid storage and consumer electronics, each with distinct degradation behaviors and performance requirements. Current modeling approaches are often domain-specific and do not support flexible fusion of semi-empirical and data-driven techniques. There is a growing need for hybrid frameworks that can generalize across use cases while preserving accuracy and computational efficiency.
\end{itemize}


\section{Problem Statement}

Although accurate estimation of the State of Health (SOH) is vital for the safe and efficient operation of lithium-ion batteries, most existing semi-empirical models exhibit significant limitations when applied in practical, real-world settings. These models typically depend on fixed parameters calibrated under controlled conditions, making them ill-suited to dynamically evolving operational environments. As usage conditions deviate from the calibration baseline, the model's predictive accuracy deteriorates, leading to cumulative estimation errors that require manual recalibration.

Furthermore, conventional semi-empirical models cannot account for nuanced degradation factors, such as the compounded effects of variable loading patterns, temperature fluctuations, and intermittent stress cycles. This lack of granularity results in oversimplified degradation profiles and reduces the models' ability to provide reliable long-term predictions.

An additional challenge lies in the absence of integrated validation mechanisms. Without real-time or automated alignment between model predictions and laboratory-based test-bench measurements, it becomes difficult to assess or improve the accuracy and credibility of the SOH estimations. These limitations affect model interpretability and hinder scalability and deployment in cloud-based platforms or across heterogeneous battery fleets.


\section{Motivation}
This research examines the semi-empirical model, which balances computational efficiency and physics-based interpretability, enhancing optimization, scalability, and lifecycle management. Though semi-empirical model balance computational efficiency and physics-based modeling, they suffer from several key limitations that affect their reliability in dynamic operating environments. The following challenges motivate the need for an enhanced framework:
\begin{itemize}
    \item Semi-empirical models rely on fixed parameter sets, assuming gradual degradation behavior using historical data.
    \item Existing semi-empirical models use static parameters and lack automated feedback or drift monitoring mechanisms.
    \item Without validation against test bench data, estimation errors accumulate over time, requiring manual recalibration.
    \item These models are not designed to integrate seamlessly into MLOps-like architectures, which hinders scalability for fleet-wide or automated deployments.
    \item No structured validation pipeline is in place for tracking model performance over time.
\end{itemize}

\section{Objectives}
The primary objective of this research is to develop an automated, scalable, and adaptive framework for estimating State of Health (SOH) that overcomes the limitations of traditional semi-empirical models. Specifically, this research aims to:
\begin{itemize}
    \item Design an optimization model to compute the correction factor that updates SOH estimates without modifying the base semi-empirical model equations.
    \item Ensure real-time correction of SOH deviations observed in the current SOH semi-empirical model.
    \item Develop an automated validation framework that continuously compares SOH estimates of a semi-empirical model against equivalent test bench conditions.
    \item Design a generalizable MLOps-like framework for deploying, validating, and maintaining semi-empirical or numerical models.
\end{itemize}

\section{Structure of the Thesis}
This thesis is structured as follows:
\begin{itemize}
    \item \textbf{Chapter \ref{chap:grundlagen}: Fundamentals and Related Work} provides an overview of the current semi-empirical SOH model, explores various correction methods, filtering approaches, hyperparameter tuning, error metrics, and cloud-scale life-cycle management architecture.
    \item \textbf{Chapter \ref{chap:main}: My Contribution} details the requirements, design, and implementation of the NMOps framework, Moving Horizon Estimation, and our algorithm, Multi-Horizon Correction Algorithm.
    \item \textbf{Chapter \ref{chap:eval}: Experimental Evaluation} presents validation experiments comparing corrected SOH with test bench ground truth and analyzes the results on various battery datasets.
    \item \textbf{Chapter \ref{chap:concl}: Conclusion} summarizes key findings and discusses future research directions, including Neural ODE/PDE-based SOH modeling.
\end{itemize}