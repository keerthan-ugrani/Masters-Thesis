Battery Management Systems (BMS) are critical for ensuring lithium-ion batteries' reliability, safety, and lifespan, especially in electric vehicles (EVs) and large-scale energy storage systems. Accurate estimation of State of Health (SOH) and State of Charge (SOC) is vital to achieving these objectives. Semi-empirical models, which combine physical insights with empirical fitting, offer a computationally efficient means of SOH estimation. However, they often suffer from poor adaptability to dynamic operating conditions, lack scalable deployment mechanisms, and lack continuous correction loops, frequently resulting in deviations from laboratory test-bench measurements.

This work proposes a novel architecture—Numerical Model Operations (NMOps), inspired by Machine Learning Operations (MLOps), to address these limitations. The framework enables the scalable deployment, monitoring, and real-time correction of semi-empirical and numerical hybrid models. Central to the architecture is the Adaptive Multi-Horizon SOH Correction Algorithm, which employs Moving Horizon Estimation (MHE) to compute optimal correction factors ($k_1$, $k_2$) over short (3-cycle) and extended (10-cycle) windows. These factors are tuned using a regularized least-squares cost function with Huber Loss to improve robustness against noise. The resulting corrected SOH values are further refined using an Extended Kalman Filter (EKF) to smooth out transient fluctuations and measurement errors, ensuring consistency across the time series.

The approach was validated using eight cells from the NASA lithium-ion battery dataset. Performance was evaluated using MSE and RMSE metrics. The proposed algorithm significantly outperformed the baseline semi-empirical model, achieving an average RMSE of 1.24\% (range: 0.38–3.61\%). The best results were observed for cell B0028 (RMSE = 0.38\%), while cell B0018 showed higher errors due to sensor noise and model deficiencies. Additionally, the NMOps framework incorporated parallel processing and demonstrated generalizability to other numerical estimation pipelines.

Overall, the NMOps architecture enables real-time feedback, scalable deployment, and automated validation, addressing the limitations of static semi-empirical approaches. Future directions include extending the framework with Neural Ordinary Differential Equations (Neural ODEs), advanced ML models like LSTM and XGBoost, MHE to parameterize temperature, voltage, and current, and distributed computing for large-scale fleet monitoring and predictive analytics.

